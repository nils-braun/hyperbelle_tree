% Summary of algorithm, introduction
One contribution was a typical tracking algorithm used in high energy physics, e.g. in the software currently written for the Belle II experiment. 
The implemented code is similar to a combinatoric Kalman filter and loosely based on the principles of a Monte Carlo search tree. 
For the sake of simplicity in this challenge, the update step of the Kalman filter (which, in real-world applications defines how well the algorithm works) was not implemented.

% steps of the algorithm
% tree search
The algorithm starts with a random hit on the outermost layer and builds up a tree using the hits on the next layer. A weight is calculated for each hit, which will be discussed below. Using only the hit with the maximal weight and those, which have a weight near to the maximal one, a new track candidate is built and the procedure is repeated until the innermost layer is processed. 

% Weight calculation
The weight calculation is the crucial part in this algorithm. 
In this example implementation, the angle difference in the xy-plane phi between two succcessive hits is sampled on training data, where the truth information is known. 
From this PDF, a probability for a given phi difference is extrated which can be used as a weight. However, in more complex applications, more advanced techniques such as multivariate methods (e.g. boosted decision trees or deep learning) can be applied.

% Final selection
The tree search may produce multiple track candidate starting with the same seed hit, so only the single best candidate according to a circular fit is stored.

As there are no background hits, the tree search is continued until all hits of the event are used.

% extrapolation
To cope with hit inefficiencies, tracks which do not include hits of all layers are fitted together pairwise and tested for their fit quality. Good combinations are merged together.

% final words
As one part of the challenge was also to write fast algorithms, the implementation was tuned afterwards. 
Caching for the hits on each layer was implemented and heavy calculations and loops were performed using \texttt{numpy} functionality.
